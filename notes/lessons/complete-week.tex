\documentclass[12pt, oneside]{article}

\usepackage[letterpaper, scale=0.89, centering]{geometry}
\usepackage{fancyhdr}
\setlength{\parindent}{0em}
\setlength{\parskip}{1em}

\usepackage{tikz}
\usetikzlibrary{automata,positioning,arrows}

\pagestyle{fancy}
\fancyhf{}
\renewcommand{\headrulewidth}{0pt}
\rfoot{\href{https://creativecommons.org/licenses/by-nc-sa/2.0/}{CC BY-NC-SA 2.0} Version \today~(\thepage)}

\usepackage{amssymb,amsmath,pifont,amsfonts,comment,enumerate,enumitem}
\usepackage{currfile,xstring,hyperref,tabularx,graphicx,wasysym}
\usepackage[labelformat=empty]{caption}
\usepackage{xcolor}
\usepackage{multicol,multirow,array,listings,tabularx,lastpage,textcomp,booktabs}

% NOTE(joe): This environment is credit @pnpo (https://tex.stackexchange.com/a/218450)
\lstnewenvironment{algorithm}[1][] %defines the algorithm listing environment
{   
    \lstset{ %this is the stype
        mathescape=true,
        frame=tB,
        numbers=left, 
        numberstyle=\tiny,
        basicstyle=\rmfamily\scriptsize, 
        keywordstyle=\color{black}\bfseries,
        keywords={,procedure, div, for, to, input, output, return, datatype, function, in, if, else, foreach, while, begin, end, }
        numbers=left,
        xleftmargin=.04\textwidth,
        #1
    }
}
{}

\newcommand\abs[1]{\lvert~#1~\rvert}
\newcommand{\st}{\mid}

\newcommand{\cmark}{\ding{51}}
\newcommand{\xmark}{\ding{55}}


\begin{document}
\begin{flushright}
    \StrBefore{\currfilename}{.}
\end{flushright}

\section*{Before we start}
If you or someone you know is suffering from food and/or housing insecurities 
there are UCSD resources here to help:

Basic Needs Office: \href{https://basicneeds.ucsd.edu/}{https://basicneeds.ucsd.edu/}

Triton Food Pantry (in the old Student Center)
is free and anonymous, and includes produce: 

\href{https://www.facebook.com/tritonfoodpantry/}{https://www.facebook.com/tritonfoodpantry/}

Mutual Aid UCSD: \href{https://mutualaiducsd.wordpress.com/}{https://mutualaiducsd.wordpress.com/}

If you find yourself in an uncomfortable situation, ask for help. 
We are committed to upholding University policies regarding nondiscrimination, sexual violence and sexual harassment.

Counseling and Psychological Services (CAPS) at 858 5343755 or \href{http://caps.ucsd.edu}{http://caps.ucsd.edu}


OPHD at (858) 534-8298, ophd@ucsd.edu , \href{http://ophd.ucsd.edu}{http://ophd.ucsd.edu}. 
CARE at Sexual Assault Resource Center at 858 5345793 sarc@ucsd.edu \href{http://care.ucsd.edu}{http://care.ucsd.edu}

\subsection*{Pandemic resilient instruction}
Fall 2021 is a transition quarter so please be patient with us as we do our best 
to serve the needs of all students while adhering to the university guidelines. 
First and foremost is the health and safety of everyone.  
Please do not come to class if you are sick or even think you might be sick.
Please reach out (minnes@eng.ucsd.edu) if you need support with extenuating circumstances.

Masks are required in class. All students who attend class must also be fully vaccinated against COVID-19
unless they have a university-approved exemption.
Campus policy requires masks and daily ``symptom screeners" for everyone and we expect all students 
to follow these rules. 


\newpage
Welcome to CSE 20: Discrete Math for Computer Science in Fall 2021! 

\section*{Themes and applications for CSE 20}
\begin{itemize}
\item {\bf Technical skepticism}: Know, select and apply appropriate computing knowledge and problem-solving techniques. 
Reason about computation and systems. 
Use mathematical techniques to solve problems. 
Determine appropriate conceptual tools to apply to new situations. 
Know when tools do not apply and try different approaches. 
Critically analyze and evaluate candidate solutions.
\item {\bf Multiple representations}: Understand, guide, shape impact of computing on society/the world. 
Connect the role of Theory CS classes to other applications (in undergraduate CS curriculum and beyond). 
Model problems using appropriate mathematical concepts.
Clearly and unambiguously communicate computational ideas using appropriate formalism. 
Translate across levels of abstraction.
\end{itemize}

{\bf Applications}: Numbers (how to represent them and use them in Computer Science), 
Recommendation systems and their roots in machine learning (with applications like Netflix),
``Under the hood" of computers (circuits, pixel color representation, data structures),
Codes and information (secret message sharing and error correction),
Bioinformatics algorithms and genomics (DNA and RNA).

\section*{Introductions}
Class website: \href{http://cseweb.ucsd.edu/classes/fa21/cse20-a}{http://cseweb.ucsd.edu/classes/fa21/cse20-a}

{\bf Pro-tip}: the URL structure is your map to finding your course website for other CSE classes.

{\bf Pro-tip}: you can use MATH109 to replace CSE20 for prerequisites and other requirements.

Instructor: Prof. Mia Minnes {\tiny{"Minnes" rhymes with Guinness}}, minnes@eng.ucsd.edu, 
\href{http://cseweb.ucsd.edu/~minnes}{http://cseweb.ucsd.edu/~minnes}

Our team: Four TAs and 10 tutors + all of you

Fill in contact info for students around you, if you'd like:
\vspace{50pt}


On a typical week: {\bf MWF} Lectures + review quizzes, {\bf T} HW due, {\bf W} Discussion, office hours, Piazza. 
Project parts will be due some weeks.

All dates are on \href{https://canvas.ucsd.edu/}{Canvas (click for link)} and details are on
 \href{https://discrete-math-for-cs.github.io/website/overview_calendar.html}{course calendar (click for link)}.

Education research: CSE 20 is participating in a project on retention and sense of community 
in UCSD majors; see \href{https://discrete-math-for-cs.github.io/files/CSInclusiveMentoringConsentFormNonCSEDataAnalysis.pdf}{research plan}. If you consent to participate in this study, no action is needed. 
If you DO NOT consent to participate in this study, or you choose to opt-out at any time during the a
cademic year, sign and submit this form to the research contact at retentionstudy@cs.ucsd.edu.


\newpage
\section*{Friday September 24}
\input{../activity-snippets/netflix-intro.tex}
\input{../activity-snippets/ratings-encoding.tex}

{\bf Conclusion}: Modeling involves choosing data types to represent and organize data

\newpage
\subsection*{Review: Week 0 Friday}
\begin{enumerate}
\item Please complete the beginning of the quarter survey \href{https://forms.gle/gvibFnNixxqcWbaU8}{https://forms.gle/gvibFnNixxqcWbaU8}
\item We want you to be familiar with class policies and procedures so you are ready to have a successful quarter. 
Please take a look at the class website http://cseweb.ucsd.edu/classes/fa21/cse20-a
and answer the questions about it on \href{http://gradescope.com}{Gradescope}.
\item Modeling: 
\begin{enumerate}
    \item {\input{../activity-snippets/quiz-ratings-tuples.tex}}
    \item {\input{../activity-snippets/quiz-ratings-count.tex}}
\end{enumerate}
\end{enumerate}
\newpage
\section*{Monday September 27}
\subsection*{Notation and prerequisites}
\input{../activity-snippets/definitions.tex}
\subsection*{Data Types: sets, $n$-tuples, and strings}
\input{../activity-snippets/data-types.tex}
\input{../activity-snippets/defining-sets.tex}
\input{../activity-snippets/rna-motivation.tex}
\input{../activity-snippets/recursive-sets-definition.tex}
%\input{../activity-snippets/set-recursive-definition.tex}
\input{../activity-snippets/set-recursive-examples.tex}
\newpage
\subsection*{Review: Week 1 Monday}
\begin{enumerate}
    \item {\input{../activity-snippets/quiz-color-rgb-definitions.tex}}
    \item {\input{../activity-snippets/quiz-set-membership.tex}}
    \item {\input{../activity-snippets/quiz-recursive-definitions.tex}}
\end{enumerate}
\newpage
\section*{Wednesday September 29}
\input{../activity-snippets/set-operations.tex}
\input{../activity-snippets/defining-functions.tex}
\input{../activity-snippets/defining-functions-ratings.tex}
\input{../activity-snippets/defining-functions-recursively.tex}
\newpage
\subsection*{Review: Week 1 Wednesday}
\begin{enumerate}
    \item {\input{../activity-snippets/quiz-ratings-set-operations.tex}}
    \item {\input{../activity-snippets/quiz-defining-functions-domain-codomain.tex}}
    \item {\input{../activity-snippets/quiz-defining-functions-recursively.tex}}
\end{enumerate}
\newpage
\section*{Friday October 1 (Zoom)}

Today's session is on Zoom, log in with your @ucsd.edu account \url{https://ucsd.zoom.us/j/97431852722} Meeting ID: 974 3185 2722

\input{../activity-snippets/why-represent-numbers.tex}
\input{../activity-snippets/base-expansion-definition.tex}
\vfill
\newpage
\input{../activity-snippets/base-expansion-examples.tex}
\input{../activity-snippets/algorithm-definition.tex}
\input{../activity-snippets/algorithm-half.tex}
\input{../activity-snippets/algorithm-log.tex}
\newpage
\input{../activity-snippets/division-algorithm.tex}
\input{../activity-snippets/base-expansion-algorithms.tex}
\newpage
\subsection*{Review: Week 1 Friday}
\begin{enumerate}
    \item {\input{../activity-snippets/quiz-division-algorithm-calculations.tex}}
    \item {\input{../activity-snippets/quiz-base-expansion-calculations.tex}}
\end{enumerate}
\newpage
\end{document}