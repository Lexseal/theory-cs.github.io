\documentclass[12pt, oneside]{article}

\usepackage[letterpaper, scale=0.89, centering]{geometry}
\usepackage{fancyhdr}
\setlength{\parindent}{0em}
\setlength{\parskip}{1em}

\usepackage{tikz}
\usetikzlibrary{automata,positioning,arrows}

\pagestyle{fancy}
\fancyhf{}
\renewcommand{\headrulewidth}{0pt}
\rfoot{\href{https://creativecommons.org/licenses/by-nc-sa/2.0/}{CC BY-NC-SA 2.0} Version \today~(\thepage)}

\usepackage{amssymb,amsmath,pifont,amsfonts,comment,enumerate,enumitem}
\usepackage{currfile,xstring,hyperref,tabularx,graphicx,wasysym}
\usepackage[labelformat=empty]{caption}
\usepackage{xcolor}
\usepackage{multicol,multirow,array,listings,tabularx,lastpage,textcomp,booktabs}

% NOTE(joe): This environment is credit @pnpo (https://tex.stackexchange.com/a/218450)
\lstnewenvironment{algorithm}[1][] %defines the algorithm listing environment
{   
    \lstset{ %this is the stype
        mathescape=true,
        frame=tB,
        numbers=left, 
        numberstyle=\tiny,
        basicstyle=\rmfamily\scriptsize, 
        keywordstyle=\color{black}\bfseries,
        keywords={,procedure, div, for, to, input, output, return, datatype, function, in, if, else, foreach, while, begin, end, }
        numbers=left,
        xleftmargin=.04\textwidth,
        #1
    }
}
{}

\newcommand\abs[1]{\lvert~#1~\rvert}
\newcommand{\st}{\mid}

\newcommand{\cmark}{\ding{51}}
\newcommand{\xmark}{\ding{55}}


\begin{document}
\begin{flushright}
    \StrBefore{\currfilename}{.}
\end{flushright}

\section*{Before we start}
If you or someone you know is suffering from food and/or housing insecurities 
there are UCSD resources here to help:

Basic Needs Office: \href{https://basicneeds.ucsd.edu/}{https://basicneeds.ucsd.edu/}

Triton Food Pantry (in the old Student Center)
is free and anonymous, and includes produce: 

\href{https://www.facebook.com/tritonfoodpantry/}{https://www.facebook.com/tritonfoodpantry/}

Mutual Aid UCSD: \href{https://mutualaiducsd.wordpress.com/}{https://mutualaiducsd.wordpress.com/}

If you find yourself in an uncomfortable situation, ask for help. 
We are committed to upholding University policies regarding nondiscrimination, sexual violence and sexual harassment.

Counseling and Psychological Services (CAPS) at 858 5343755 or \href{http://caps.ucsd.edu}{http://caps.ucsd.edu}


OPHD at (858) 534-8298, ophd@ucsd.edu , \href{http://ophd.ucsd.edu}{http://ophd.ucsd.edu}. 
CARE at Sexual Assault Resource Center at 858 5345793 sarc@ucsd.edu \href{http://care.ucsd.edu}{http://care.ucsd.edu}

\subsection*{Pandemic resilient instruction}
Fall 2021 is a transition quarter so please be patient with us as we do our best 
to serve the needs of all students while adhering to the university guidelines. 
First and foremost is the health and safety of everyone.  
Please do not come to class if you are sick or even think you might be sick.
Please reach out (minnes@eng.ucsd.edu) if you need support with extenuating circumstances.

Masks are required in class. All students who attend class must also be fully vaccinated against COVID-19
unless they have a university-approved exemption.
Campus policy requires masks and daily ``symptom screeners" for everyone and we expect all students 
to follow these rules. 


\newpage
Welcome to CSE 20: Discrete Math for Computer Science in Fall 2021! 

\section*{Themes and applications for CSE 20}
\begin{itemize}
\item {\bf Technical skepticism}: Know, select and apply appropriate computing knowledge and problem-solving techniques. 
Reason about computation and systems. 
Use mathematical techniques to solve problems. 
Determine appropriate conceptual tools to apply to new situations. 
Know when tools do not apply and try different approaches. 
Critically analyze and evaluate candidate solutions.
\item {\bf Multiple representations}: Understand, guide, shape impact of computing on society/the world. 
Connect the role of Theory CS classes to other applications (in undergraduate CS curriculum and beyond). 
Model problems using appropriate mathematical concepts.
Clearly and unambiguously communicate computational ideas using appropriate formalism. 
Translate across levels of abstraction.
\end{itemize}

{\bf Applications}: Numbers (how to represent them and use them in Computer Science), 
Recommendation systems and their roots in machine learning (with applications like Netflix),
``Under the hood" of computers (circuits, pixel color representation, data structures),
Codes and information (secret message sharing and error correction),
Bioinformatics algorithms and genomics (DNA and RNA).

\section*{Introductions}
Class website: \href{http://cseweb.ucsd.edu/classes/fa21/cse20-a}{http://cseweb.ucsd.edu/classes/fa21/cse20-a}

{\bf Pro-tip}: the URL structure is your map to finding your course website for other CSE classes.

{\bf Pro-tip}: you can use MATH109 to replace CSE20 for prerequisites and other requirements.

Instructor: Prof. Mia Minnes {\tiny{"Minnes" rhymes with Guinness}}, minnes@eng.ucsd.edu, 
\href{http://cseweb.ucsd.edu/~minnes}{http://cseweb.ucsd.edu/~minnes}

Our team: Four TAs and 10 tutors + all of you

Fill in contact info for students around you, if you'd like:
\vspace{50pt}


On a typical week: {\bf MWF} Lectures + review quizzes, {\bf T} HW due, {\bf W} Discussion, office hours, Piazza. 
Project parts will be due some weeks.

All dates are on \href{https://canvas.ucsd.edu/}{Canvas (click for link)} and details are on
 \href{https://discrete-math-for-cs.github.io/website/overview_calendar.html}{course calendar (click for link)}.

Education research: CSE 20 is participating in a project on retention and sense of community 
in UCSD majors; see \href{https://discrete-math-for-cs.github.io/files/CSInclusiveMentoringConsentFormNonCSEDataAnalysis.pdf}{research plan}. If you consent to participate in this study, no action is needed. 
If you DO NOT consent to participate in this study, or you choose to opt-out at any time during the a
cademic year, sign and submit this form to the research contact at retentionstudy@cs.ucsd.edu.


\newpage
\section*{Friday September 24}
\input{../activity-snippets/netflix-intro.tex}
\input{../activity-snippets/ratings-encoding.tex}

{\bf Conclusion}: Modeling involves choosing data types to represent and organize data

\newpage
\subsection*{Review: Week 0 Friday}
\begin{enumerate}
\item Please complete the beginning of the quarter survey \href{https://forms.gle/gvibFnNixxqcWbaU8}{https://forms.gle/gvibFnNixxqcWbaU8}
\item We want you to be familiar with class policies and procedures so you are ready to have a successful quarter. 
Please take a look at the class website http://cseweb.ucsd.edu/classes/fa21/cse20-a
and answer the questions about it on \href{http://gradescope.com}{Gradescope}.
\item Modeling: 
\begin{enumerate}
    \item {\input{../activity-snippets/quiz-ratings-tuples.tex}}
    \item {\input{../activity-snippets/quiz-ratings-count.tex}}
\end{enumerate}
\end{enumerate}
\newpage
\section*{Monday September 27}
\subsection*{Notation and prerequisites}
\input{../activity-snippets/definitions.tex}
\subsection*{Data Types: sets, $n$-tuples, and strings}
\input{../activity-snippets/data-types.tex}
\input{../activity-snippets/defining-sets.tex}
\input{../activity-snippets/rna-motivation.tex}
\input{../activity-snippets/recursive-sets-definition.tex}
%\input{../activity-snippets/set-recursive-definition.tex}
\input{../activity-snippets/set-recursive-examples.tex}
\newpage
\subsection*{Review: Week 1 Monday}
\begin{enumerate}
    \item {\input{../activity-snippets/quiz-color-rgb-definitions.tex}}
    \item {\input{../activity-snippets/quiz-set-membership.tex}}
    \item {\input{../activity-snippets/quiz-recursive-definitions.tex}}
\end{enumerate}
\newpage
\section*{Wednesday September 29}
\input{../activity-snippets/set-operations.tex}
\input{../activity-snippets/defining-functions.tex}
\input{../activity-snippets/defining-functions-ratings.tex}
\input{../activity-snippets/defining-functions-recursively.tex}
\newpage
\subsection*{Review: Week 1 Wednesday}
\begin{enumerate}
    \item {\input{../activity-snippets/quiz-ratings-set-operations.tex}}
    \item {\input{../activity-snippets/quiz-defining-functions-domain-codomain.tex}}
    \item {\input{../activity-snippets/quiz-defining-functions-recursively.tex}}
\end{enumerate}
\newpage
\section*{Friday October 1 (Zoom)}

Today's session is on Zoom, log in with your @ucsd.edu account \url{https://ucsd.zoom.us/j/97431852722} Meeting ID: 974 3185 2722

\input{../activity-snippets/why-represent-numbers.tex}
\input{../activity-snippets/base-expansion-definition.tex}
\vfill
\newpage
\input{../activity-snippets/base-expansion-examples.tex}
\input{../activity-snippets/algorithm-definition.tex}
\input{../activity-snippets/algorithm-half.tex}
\input{../activity-snippets/algorithm-log.tex}
\newpage
\input{../activity-snippets/division-algorithm.tex}
\input{../activity-snippets/base-expansion-algorithms.tex}
\newpage
\subsection*{Review: Week 1 Friday}
\begin{enumerate}
    \item {\input{../activity-snippets/quiz-division-algorithm-calculations.tex}}
    \item {\input{../activity-snippets/quiz-base-expansion-calculations.tex}}
\end{enumerate}
\newpage

\section*{Monday November 29}
\input{../activity-snippets/equivalence-relations-partitions.tex}

Last time, we saw that partitions associated to equivalence relations
were useful in the context of modular arithmetic.
Today we'll look at a different application.

\input{../activity-snippets/equivalence-relations-examples-ratings.tex}
\newpage
\section*{Review}
\begin{enumerate}
    \item Select all and only the partitions of $\{1,2,3,4,5\}$ from the sets below.
    \begin{enumerate}
    \item $\{1,2,3,4,5\}$
    \item $\{\{1,2,3,4,5\}\}$
    \item $\{\{1\},\{2\},\{3\},\{4\},\{5\}\}$
    \item $\{ \{1\}, \{2,3\}, \{4\} \}$
    \item $\{ \{\emptyset, 1, 2\}, \{3,4,5\}\}$
    \end{enumerate}
    \item \hspace{1in}\\ \input{../activity-snippets/quiz-equivalence-class-types.tex}
\end{enumerate}
\newpage

\section*{Wednesday December 1}
\input{../activity-snippets/netflix-clustering-scenario.tex}
\newpage

\section*{Looking forward}

\subsection*{Tips for future classes from the CSE 20 TAs and tutors}
\begin{itemize}
\item In class
\begin{itemize}
\item Go to class
\item Show up to class early because sometimes seats get taken/ the classroom gets full and then you have to sit on the floor
\item There's usually a space for skateboards/longboards/eboards to go at the front or rear of the lecture hall 
\item If you have a flask water bottle please ensure that its secured during a lecture and it cannot fall - putting on the floor often leads to it falling since people sometimes cross your seats.
\item Take notes - it's much faster and more effective to note-take in class than watch recordings after, particularly if you do so longhand
\item Resist the urge to sit in the back. You will be able to focus much better sitting near the front, where there are fewer screens in front of you to distract from the lecture content
\item If you bring your laptop to class to take notes / access class materials, sit towards the back of the room to minimize distractions for people sitting behind you!
\item On zoom it's easy to just type a question out in chat, it might be a little more awkward to do so in person, but it is definitely worth it. Don't feel like you should already know what's being covered
\item Always check you have your iclicker\footnote{iclickers are used in many classes to encourage active participation in class. They're remotes that allow you to respond to multiple choice questions and the instructor can show a histogram of responses in real time.} on you. Just keep it in your backpack permanently. That way you can never forget it. 
\item Don't be afraid to talk to the people next to you during group discussions. Odds are they're as nervous as you are, and you can all benefit from sharing your thoughts and understanding of the material 
\item Certain classes will podcast the lectures, just like Zoom archives lecture recordings, at podcast.ucsd.edu
\item If they aren't podcasted, and you want to record lectures, ask your professor for consent first
\end{itemize}
\item Office hours, tutor hours, and the CSE building
\begin{itemize}
\item Office hours are a good place to hang out and get work done while being able to ask questions as they come up 
\item Office hour attendance is typically much busier in person (and confined to the space in the room)
\item Get to know the CSE building: CSE B260, basement labs, office hours rooms
\item Know how to get in to the building after-hours
\end{itemize}
\item Libraries and on-campus resources
\begin{itemize}
\item Look up what library floors are for what, how to book rooms: East wing of Geisel is open 24/7 (they might ask to see an ID if you stay late), East Wing of Geisel has chess boards and jigsaw puzzles, study pods on the 8th floor, 
free computers/wifi
\item Know Biomed exists and is usually less crowded
\item Most libraries allow you to borrow whiteboards and markers (also laptops, tablets, microphones, and other cool stuff) for 24 hours
\item Take advantage of Dine with a prof / Coffee with a prof program. It's legit free food / coffee once per quarter. 
\item When planning out your daily schedule, think about where classes are, how much time will they take, are their places to eat nearby and how you can schedule social time with friends to nearby areas 
\item Take into account the distances between classes if they are back to back
\end{itemize} 
\item Final exams
\begin{itemize}
\item What are 8am finals? Basically in-person exams are different
\item Don't forget your university card during exams
\item Blue books for exams (what they are, where to get them) 
\item Seating assignments for exams and go early to make sure you're in the right place (check the exits to make sure you're reading it the right way) 
\item Know where your exam is being held (find it on a map at least a day beforehand). Finals are often in strange places that take a while to find 
\end{itemize}
\end{itemize}

\subsection*{CSE department course numbering system}

{\bf Lower division}

\begin{itemize}
\setlength\itemsep{-2pt}
\item CSE 12, Basic Data Structures and Object-Oriented Programming 
\item CSE 15L (2 units), Software Tools and Technique Laboratory 
\item CSE 20 or Math 15A, Introduction to Discrete Mathematics
\item CSE 21 Mathematics for Algorithms and Systems
\item CSE 30, Computer Organization \& Systems Programming
\end{itemize}

{\bf Upper division}

\begin{itemize}
\setlength\itemsep{-2pt}
\item Advanced Data Structures and Programming: CSE 100
\item Theory and Algorithms: CSE 101, CSE 105
\item Software Engineering: CSE 110, CSE 112
\item Systems/Networks: CSE 120 or CSE 123 or CSE 124
\item Programming Languages /Databases: CSE 130 or CSE 132A
\item Security/Cryptography: CSE 107 or CSE 127
\item AI / Machine Learning/ Vision/ Graphics:
CSE 150A or CSE 151A or CSE 151B or CSE 152A or CSE 158 or CSE 167
\item Hardware / Architecture: CSE 140/ CSE 140L Components and Design Techniques for Digital Systems Architecture, CSE 141 / CSE 141L Introduction to Computer Architecture and CSE 141L Project in Computer Architecture (2 units), CSE 142 / CSE 142L Comp Arch Software Perspective
\end{itemize}

\subsection*{Review}
\begin{enumerate}
    \item \hspace{1in}\\ \input{../activity-snippets/quiz-binary-relation-ratings.tex}
    \item \hspace{1in}\\ \input{../activity-snippets/quiz-clustering-ratings.tex} 
\end{enumerate}

\newpage
\section*{Friday December 3}
\input{../activity-snippets/base-expansion-final-review.tex}
\newpage
\input{../activity-snippets/set-construction-final-review.tex}
\newpage
\input{../activity-snippets/set-operations-final-review.tex}
\newpage
\input{../activity-snippets/function-properties-final-review.tex}
\newpage
\input{../activity-snippets/cardinality-final-review.tex}
\newpage
\input{../activity-snippets/modular-arithmetic-final-review.tex}
\newpage
\subsection*{Review}
\begin{enumerate}
    \item \hspace{1in}\\  Please complete the CAPE and TA evaluations.  Once 
    you have done so complete the custom feedback form for this quarter: 
    \url{https://forms.gle/pbYWSRDP2znkciM46}
    
    Then, (we're using the honor system here), write out the statement
    ``I have completed the end  of quarter evaluations"  and you'll receive credit 
    for  this  question.
\end{enumerate}
\newpage

% Today's session is on Zoom, log in with your @ucsd.edu account \url{https://ucsd.zoom.us/j/97431852722} Meeting ID: 974 3185 2722

\section*{Monday October 4}
\input{../activity-snippets/base-expansion-review.tex}
\input{../activity-snippets/base-conversion-algorithm.tex}
\input{../activity-snippets/fixed-width-definition.tex}
\input{../activity-snippets/fixed-width-example.tex}
\input{../activity-snippets/fixed-width-fractional-definition.tex}

\newpage
\subsection*{Review: Week 2 Monday}
\begin{enumerate}
    \item {\input{../activity-snippets/quiz-fixed-width-expansions.tex}}
    \item {\input{../activity-snippets/quiz-expansions-properties.tex}}
\end{enumerate}
\newpage
\section*{Wednesday October 6}
\input{../activity-snippets/expansion-summary.tex}
\input{../activity-snippets/negative-int-expansions.tex}
\newpage
\input{../activity-snippets/calculating-2s-complement.tex}
\input{../activity-snippets/representing-zero.tex}
\newpage
\input{../activity-snippets/fixed-width-addition.tex}
\newpage
\subsection*{Review: Week 2 Wednesday}
\begin{enumerate}
    \item {\input{../activity-snippets/quiz-representing-negatives.tex}}
    \item {\input{../activity-snippets/quiz-fixed-width-addition.tex}}
\end{enumerate}
\newpage
\section*{Friday October 8}
\input{../activity-snippets/circuits-basics.tex}
\input{../activity-snippets/logic-gates-definitions.tex}
\input{../activity-snippets/digital-circuits-basic-examples.tex}
\newpage
\input{../activity-snippets/half-adder-circuit.tex}
\newpage
\input{../activity-snippets/two-bit-adder-circuit.tex}
\newpage
\subsection*{Review: Week 2 Friday}
\begin{enumerate}
    \item {\input{../activity-snippets/quiz-circuit-tracing.tex}}
    \item {\input{../activity-snippets/quiz-circuit-implementing-operation.tex}}
\end{enumerate}

\newpage

\section*{Monday October 11}
\input{../activity-snippets/logical-operators.tex}
\input{../activity-snippets/logical-operators-truth-tables.tex}
\input{../activity-snippets/logical-operators-example-truth-table.tex}
\newpage
\input{../activity-snippets/truth-table-to-compound-proposition.tex}
\input{../activity-snippets/dnf-cnf-definition.tex}
\input{../activity-snippets/dnf-cnf-example.tex}
\newpage
\subsection*{Review: Week 3 Monday}
\begin{enumerate}
    \item \input{../activity-snippets/quiz-circuit-tracing-with-or.tex}
    \item \input{../activity-snippets/quiz-cnf-dnf.tex}
\end{enumerate}
\newpage
\section*{Wednesday October 13}
\input{../activity-snippets/compound-proposition-definitions.tex}
\input{../activity-snippets/logical-equivalence.tex}
\vfill
\input{../activity-snippets/tautology-contradiction-contingency-examples.tex}
\vfill
\input{../activity-snippets/logical-equivalence-extra-example.tex}
\vfill
\input{../activity-snippets/logical-operators-full-truth-table.tex}
\input{../activity-snippets/hypothesis-conclusion.tex}
\input{../activity-snippets/converse-inverse-contrapositive.tex}
\vfill
\input{../activity-snippets/compound-propositions-recursive-definition.tex}
\input{../activity-snippets/compound-propositions-precedence.tex}
\newpage
\input{../activity-snippets/logical-equivalence-identities.tex}
\newpage
\subsection*{Review: Week 3 Wednesday}
\begin{enumerate}
    \item \input{../activity-snippets/quiz-truth-values-or-and.tex}
    \item \input{../activity-snippets/quiz-truth-values-conditional.tex}
\end{enumerate}
\newpage
\section*{Friday October 15}
\input{../activity-snippets/logical-operators-english-synonyms.tex}
\newpage
\newpage
\input{../activity-snippets/compound-propositions-translation.tex}
\newpage
\input{../activity-snippets/consistency-def.tex}
\input{../activity-snippets/consistency-example.tex}
\newpage
\subsection*{Review: Week 3 Friday}
\begin{enumerate}
    \item \input{../activity-snippets/quiz-compound-propositions-translation.tex}
    \item \input{../activity-snippets/quiz-consistency.tex}
\end{enumerate}

\newpage%4


\section*{Monday October 18}
\input{../activity-snippets/algorithm-redundancy.tex}
\newpage
\input{../activity-snippets/cartesian-product-definition.tex}
\input{../activity-snippets/algorithm-rna-mutation-insertion-deletion.tex}
\input{../activity-snippets/rna-mutation-insertion-deletion-example.tex}
\newpage

\newpage
\subsection*{Review}
\begin{enumerate}
\item \hspace{1in}\\ \input{../activity-snippets/quiz-building-circuit.tex}
\item \hspace{1in}\\ \input{../activity-snippets/quiz-redundancy-algorithm.tex}
\item \hspace{1in}\\ \input{../activity-snippets/quiz-rna-mutation-insertion-deletion.tex}
\end{enumerate}

\newpage
\section*{Wednesday October 20}
\input{../activity-snippets/predicate-definition.tex}
\input{../activity-snippets/predicate-examples-finite-domain.tex}
\vfill
\input{../activity-snippets/predicate-truth-set-definition.tex}
\input{../activity-snippets/predicate-truth-set-example.tex}
\newpage
\input{../activity-snippets/quantification-definition.tex}
\input{../activity-snippets/quantification-logical-equivalence.tex}
\input{../activity-snippets/quantification-examples-finite-domain.tex}
\input{../activity-snippets/predicate-rna-example.tex}
\newpage
\subsection*{Review}
\begin{enumerate}
\item \hspace{1in}\\ \input{../activity-snippets/quiz-predicates-finite-domain.tex}
\item \hspace{1in}\\ \input{../activity-snippets/quiz-predicates.tex}
\item \hspace{1in}\\ \input{../activity-snippets/quiz-predicates-rna.tex}
\end{enumerate}

\newpage
\section*{Friday October 22}
\input{../activity-snippets/rna-rnalen-basecount-definitions.tex}
\input{../activity-snippets/predicates-example-rnalen-basecount.tex}
\input{../activity-snippets/predicates-projecting-example-rna-basecount.tex}
\input{../activity-snippets/predicate-notation.tex}
\newpage
\input{../activity-snippets/nested-quantifiers.tex}
\input{../activity-snippets/alternating-quantifiers-strategies-rna-examples.tex}
\newpage
\subsection*{Review}
\begin{enumerate}
\item \hspace{1in}\\\input{../activity-snippets/quiz-predicates-alternating-quantifiers-rnalen.tex}
\item \hspace{1in}\\\input{../activity-snippets/quiz-predicates-alternating-quantifiers-basecount.tex}
\end{enumerate}


\newpage

\section*{Monday October 25}
\subsection*{Proof strategies}
\input{../activity-snippets/proof-strategies-road-map.tex}
\input{../activity-snippets/proof-strategies-quantification-finite-domain.tex}
\input{../activity-snippets/proof-strategy-universal-exhaustion.tex}
\input{../activity-snippets/proof-strategy-universal-generalization.tex}
\newpage
\input{../activity-snippets/sets-equality-subset-definition.tex}
\input{../activity-snippets/proof-strategies-conditionals.tex}
\input{../activity-snippets/proof-strategies-proof-by-cases.tex}
\input{../activity-snippets/proof-strategies-ands.tex}
\input{../activity-snippets/sets-proof-strategies.tex}
\input{../activity-snippets/sets-equality-example.tex}
\newpage
\input{../activity-snippets/sets-basic-proofs.tex}
\vfill
\input{../activity-snippets/proofs-signposting.tex}
\newpage
\subsection*{Review}
\begin{enumerate}
\item \hspace{1in}\\ \input{../activity-snippets/quiz-translating-counting-quantifiers.tex}
\item \hspace{1in}\\ \input{../activity-snippets/quiz-sets-claims-subset-equality.tex}
\item We want to hear how the term and this class are going for you.
Please complete the midquarter feedback form: \href{https://forms.gle/w3D7ifAWnD5sWwHf9}{https://forms.gle/w3D7ifAWnD5sWwHf9}
\end{enumerate}

\newpage
\section*{Wednesday October 27}
\input{../activity-snippets/set-operations-union-intersection-powerset.tex}
\newpage
\input{../activity-snippets/sets-basic-proofs-operations.tex}
\newpage
\subsection*{Review}
\begin{enumerate}
\item \hspace{1in}\\ \input{../activity-snippets/quiz-sets-claims.tex}
\item \hspace{1in}\\ \input{../activity-snippets/quiz-sets-proof-strategies.tex}
\end{enumerate}

\newpage
\section*{Friday October 29}
\subsection*{Facts about numbers}
\input{../activity-snippets/numbers-facts.tex}
\subsection*{Factoring}
\input{../activity-snippets/factoring-definition.tex}
\input{../activity-snippets/factoring-translation-examples.tex}
\input{../activity-snippets/factoring-basic-claims.tex}
\newpage
\input{../activity-snippets/factoring-basic-claims-continued.tex}
\input{../activity-snippets/factoring-even-odd.tex}
\input{../activity-snippets/prime-number-definition.tex}
\input{../activity-snippets/primes-basic-claims.tex}
\newpage
\subsection*{Review}
\begin{enumerate}
    \item \hspace{1in}\\ \input{../activity-snippets/quiz-factoring-quantifiers.tex}
    \item \hspace{1in}\\ \input{../activity-snippets/quiz-prime-formalizing-definition.tex}
\end{enumerate}


\newpage

\section*{Monday November 1}

Today's session is on Zoom, log in with your @ucsd.edu account https://ucsd.zoom.us/j/97431852722
Meeting ID: 974 3185 2722

\input{../activity-snippets/rna-rnalen-basecount-definitions.tex}

At this point, we've seen the proof strategies
\begin{multicols}{2}
    \begin{itemize}
        \item A {\bf counterexample} to prove that  $\forall x P(x)$ is {\bf false}.
        \item  A {\bf witness} to prove that  $\exists x P(x)$ is {\bf true}.
        \item {\bf Proof of universal by exhaustion} to prove that $\forall x \, P(x)$
    is true when $P$ has a finite domain
        \item  {\bf Proof by universal generalization} to prove that $\forall x \, P(x)$
    is true using an arbitrary element of the domain.
        \item To  prove  that $\exists x P(x)$ is {\bf false}, write the universal statement that is 
        logically equivalent to its negation and then prove it true using universal generalization.
        \item To prove that $p \land q$ is true, have two subgoals: 
        subgoal (1) prove $p$ is  true; and, subgoal (2) prove $q$ is true. To prove that $p \land q$ is false, it's enough to prove that $p$ is false.
     To prove that $p \land q$ is false, it's enough to prove that $q$ is false.
        \item Proof of conditional by {\bf direct proof}
        \item Proof of conditional by {\bf contrapositive proof}
        \item Proof of disjuction using equivalent conditional: To prove that the 
        disjunction $p \lor q$ is true, we can rewrite it equivalently as $\lnot p \to q$ and
        then use direct proof or contrapositive proof.
        \item {\bf Proof by cases}.
    \end{itemize}
\end{multicols}
\newpage
\input{../activity-snippets/alternating-quantifiers-proofs-rna-examples.tex}
\newpage
\input{../activity-snippets/structural-induction-motivating-example-rna.tex}
\input{../activity-snippets/proof-strategies-structural-induction.tex}
\newpage
\input{../activity-snippets/structural-induction-example-rnalen-basecount.tex}
\newpage
\subsection*{Review}
\input{../activity-snippets/quiz-basecount-rnalen-induction.tex}

\newpage
\section*{Wednesday November 3}
\input{../activity-snippets/proofs-signposting-kinds-of-claims.tex}
\input{../activity-snippets/structural-induction-example-robot-grid.tex}
\newpage
\input{../activity-snippets/structural-induction-example-sum-of-powers.tex}
\vfill
\input{../activity-snippets/proof-strategy-mathematical-induction.tex}
\newpage
\subsection*{Review}
\begin{enumerate}
\item \hspace{1in}\\ \input{../activity-snippets/quiz-robot-grid.tex}
\item \hspace{1in}\\ \input{../activity-snippets/quiz-comparing-structural-mathematical-induction.tex}
\newpage
\item \hspace{1in}\\ \input{../activity-snippets/quiz-exponential-factorial.tex}
\end{enumerate}

\newpage
\section*{Friday November 5}
\input{../activity-snippets/linked-lists-definition.tex}
\input{../activity-snippets/linked-lists-examples.tex}
\input{../activity-snippets/linked-list-length-definition.tex}
\vspace{50pt}
\input{../activity-snippets/linked-lists-prepend-definition.tex}
\vspace{50pt}
\input{../activity-snippets/linked-list-append-definition.tex}
\vspace{50pt}
\newpage
\input{../activity-snippets/linked-list-append-length-claim-proof.tex}
\newpage
\input{../activity-snippets/linked-list-example-each-length.tex}
\newpage
\subsection*{Review}
\input{../activity-snippets/quiz-linked-list-definitions.tex}


\newpage

\section*{Monday November 8}
\subsection*{Visualizing induction}
\input{../activity-snippets/induction-dominos.tex}
\input{../activity-snippets/proof-strategy-mathematical-induction.tex}
\input{../activity-snippets/proof-strategy-strong-induction.tex}
\newpage
\input{../activity-snippets/binary-expansions-exist-proof.tex}

\subsubsection*{Representing positive integers with primes}
\input{../activity-snippets/fundamental-theorem-proof.tex}
\subsubsection*{Sending old-fashioned mail with postage stamps}
\input{../activity-snippets/strong-induction-making-change-proof-idea.tex}
\newpage
\subsection*{Review}
\begin{enumerate}
    \item \hspace{1in} \\ \input{../activity-snippets/quiz-binary-expansions-exist-invalid-proof.tex}
    \item \hspace{1in}\\ \input{../activity-snippets/quiz-prime-factorization.tex}
    \newpage
    \item \input{../activity-snippets/quiz-making-change-proof-two-ways.tex}
\end{enumerate}

\newpage
\section*{Wednesday November 10}

\subsubsection*{Finding a winning strategy for a game}
\input{../activity-snippets/strong-induction-nim.tex}
\newpage
\input{../activity-snippets/proof-strategy-proof-by-contradiction.tex}
\subsection*{Least and greatest}
\input{../activity-snippets/least-greatest-proofs.tex}

\newpage
\subsection*{Review}
\begin{enumerate}
\item \hspace{1in}\\ \input{../activity-snippets/quiz-nim.tex}
\item \hspace{1in}\\ \input{../activity-snippets/quiz-no-greatest-prime.tex}
\item \hspace{1in}\\ \input{../activity-snippets/quiz-choosing-proof-strategy.tex}
\end{enumerate}

\newpage
\section*{Friday November 12}

\input{../activity-snippets/gcd-definition.tex}
\input{../activity-snippets/gcd-examples.tex}
\input{../activity-snippets/gcd-basic-claims.tex}
\input{../activity-snippets/gcd-lemma-relatively-prime.tex}

\newpage
\subsection*{Sets of numbers}

We've seen multiple representations of the set of positive integers
(using base expansions and using prime factorization). Now we're 
going to expand our attention to other sets of numbers as well.
\input{../activity-snippets/rational-numbers-definition.tex}
\input{../activity-snippets/sets-numbers-subsets.tex}
\input{../activity-snippets/proof-by-contradiction-irrational.tex}

\newpage
\subsection*{Review}
\begin{enumerate}
\item \hspace{1in}\\ \input{../activity-snippets/quiz-calculating-gcd.tex}
\newpage
\item \hspace{1in}\\ \input{../activity-snippets/quiz-odd-even-proofs.tex}
\end{enumerate}

\newpage%8


\section*{Monday November 15}

\input{../activity-snippets/sets-numbers-subsets.tex}
\input{../activity-snippets/finite-sets-definition.tex}
\input{../activity-snippets/cardinality-motivation.tex}
\input{../activity-snippets/cardinality-rationale-for-functions.tex}
\input{../activity-snippets/musical-chairs-analogy.tex}
\newpage
\input{../activity-snippets/well-defined-functions.tex}
\input{../activity-snippets/injective-function-definition.tex}
\input{../activity-snippets/injective-functions-visually.tex}
\input{../activity-snippets/cardinality-lower-bound-definition.tex}
\input{../activity-snippets/injective-cardinality-musical-chairs.tex}
\input{../activity-snippets/rna-injective-cardinality.tex}
\input{../activity-snippets/surjective-function-definition.tex}
\input{../activity-snippets/surjective-functions-visually.tex}
\input{../activity-snippets/cardinality-upper-bound-definition.tex}
\input{../activity-snippets/surjective-cardinality-musical-chairs.tex}
\newpage
\input{../activity-snippets/rna-surjective-cardinality.tex}
\input{../activity-snippets/bijection-definition.tex}
\subsection*{Cardinality of sets}
\input{../activity-snippets/cardinality-definition.tex}
\input{../activity-snippets/cardinality-caution.tex}
\newpage
\section*{Review}
\begin{enumerate}
    \item \hspace{1in}\\ \input{../activity-snippets/quiz-finite-sets.tex}
    \item \hspace{1in}\\ \input{../activity-snippets/quiz-injective-surjective.tex}
    \newpage
    \item \hspace{1in}\\ \input{../activity-snippets/quiz-cardinality-witnessing-functions-q1.tex} 
\end{enumerate}
\newpage

\section*{Wednesday November 17}
\input{../activity-snippets/cardinality-properties.tex}
\vspace{100pt}
\input{../activity-snippets/cantor-schroder-bernstein-theorem.tex}
\newpage
\input{../activity-snippets/countably-infinite-definition.tex}
\input{../activity-snippets/countably-infinite-examples-sets-of-numbers.tex}
\input{../activity-snippets/countably-infinite-examples-other-sets.tex}
\newpage
\subsection*{Review}
\begin{enumerate}
    \item \input{../activity-snippets/quiz-cardinality-witnessing-functions-q2.tex} 
    \item \hspace{1in}\\ \input{../activity-snippets/quiz-rationals-proofs.tex}
\end{enumerate}

\newpage
\section*{Friday November 19}
\subsection*{Cardinality categories}
\input{../activity-snippets/cardinality-categories.tex}
\input{../activity-snippets/cardinality-countability-lemmas.tex}
\newpage
\subsection*{Are there always *bigger* sets?}
\input{../activity-snippets/cardinality-power-sets.tex}
\newpage
\subsection*{Countable vs.\ uncountable: sets of numbers}
\input{../activity-snippets/cardinality-rationals-reals.tex}
\subsection*{Other examples of uncountable sets}
\input{../activity-snippets/cardinality-uncountable-examples.tex}
\newpage
\subsection*{Review}
\begin{enumerate}
    \item \hspace{1in}\\ \input{../activity-snippets/quiz-diagonalization.tex}
    \item \hspace{1in}\\ \input{../activity-snippets/quiz-cardinality-classifying.tex}
\end{enumerate}
\newpage%8


\section*{Monday November 22}

\input{../activity-snippets/binary-relation-definition.tex}
\input{../activity-snippets/relations-as-graphs.tex}
\input{../activity-snippets/binary-relation-examples.tex}
\newpage
\input{../activity-snippets/reflexive-relation-definition.tex}
\input{../activity-snippets/reflexive-relation-informally.tex}
\vfill
\input{../activity-snippets/symmetric-relation-definition.tex}
\input{../activity-snippets/symmetric-relation-informally.tex}
\vfill
\input{../activity-snippets/transitive-relation-definition.tex}
\input{../activity-snippets/transitive-relation-informally.tex}
\vfill
\input{../activity-snippets/antisymmetric-relation-definition.tex}
\input{../activity-snippets/antisymmetric-relation-informally.tex}
\vfill
\newpage
\input{../activity-snippets/binary-relation-properties-examples.tex}
\newpage
\input{../activity-snippets/equivalence-relation-definition.tex}
\input{../activity-snippets/partial-order-definition.tex}
\input{../activity-snippets/hasse-diagram-definition.tex}
\input{../activity-snippets/hasse-diagram-example.tex}
\vfill

{\it Summary}: binary relations can be useful for organizing elements in a domain. 
Some binary relations have special properties that make them act like some familiar relations.
Equivalence relations (reflexive, symmetric, transitive binary relations) ``act like'' equals.
Partial orders (reflexive, antisymmetric, transitive binary relations) ``act like'' less than or equals to.
\newpage
\section*{Review}
\begin{enumerate}
    \item \hspace{1in}\\ \input{../activity-snippets/quiz-binary-relation-same-size.tex} 
    \item \hspace{1in}\\ \input{../activity-snippets/quiz-binary-relation-divisibility.tex}
    \item \hspace{1in}\\ \input{../activity-snippets/quiz-partial-order-hasse.tex}
\end{enumerate}
\newpage

\section*{Wednesday November 24}
\subsection*{Exploring equivalence relations}
\input{../activity-snippets/partition-definition.tex}
\input{../activity-snippets/equivalence-class-definition.tex}
\input{../activity-snippets/partitions-equivalence-classes.tex}
\input{../activity-snippets/congruence-classes-mod-four.tex}
\input{../activity-snippets/modular-arithmetic-motivation.tex}
\input{../activity-snippets/congruence-mod-n-lemma.tex}
\input{../activity-snippets/modular-arithmetic-cycling-examples.tex}
\input{../activity-snippets/modular-arithmetic.tex}
\newpage
\input{../activity-snippets/diffie-helman.tex}
\newpage
\subsection*{Review}
\begin{enumerate}
    \item \hspace{1in}\\ \input{../activity-snippets/quiz-equivalence-relation-properties-proof.tex}
    \item \hspace{1in}\\ \input{../activity-snippets/quiz-modular-exponentiation.tex}
\end{enumerate}

\newpage
\section*{Friday November 26}

No class, in observance of Thanksgiving holiday.
\newpage
\end{document}