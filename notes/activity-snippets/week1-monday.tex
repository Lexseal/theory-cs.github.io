%! app: regular-languages
%! outcome: Define decision problem

We will use vocabulary that should be familiar from your discrete 
math and introduction to proofs classes.  Some of the notation conventions may 
be a bit different: we will use the notation from this class' textbook\footnote{Page references are to 
the 3rd edition (International) of Siper's Introduction to the Theory of Computation,
available at the campus bookstore for under \$20. Copies of the book are also available 
for those who can't access the book
to borrow from the course instructor, while supplies last (minnes@eng.ucsd.edu)}.

Write out in words the meaning of the symbols below: 
\[
    \{ a,b, c\}
\]

\phantom{The set whose elements are $a$, $b$, and $c$}

\[
    | \{a, b, a \} | = 2
\]

\phantom{The number of elements in the set $\{a,b,a\}$ is $2$.}

\[
    | aba | = 3
\]

\phantom{The length of the string $aba$ is $3$.}

\[
    (a, 3, 2, b, b)
\]

\phantom{The $5$-tuple whose first components is $a$, second component 
is $3$, third component is $2$, fourth component is $b$, and fifth component is $b$.}



\begin{center}
    \begin{tabular}{|p{2in}cp{4in}|}
    \hline 
    Term & Typical symbol & Meaning \\
    \hline\hline
    Alphabet & $\Sigma$, $\Gamma$ & A non-empty finite set	 \\ \hline
    Symbol over $\Sigma$  & $\sigma$, $b$, $x$ & An element of the alphabet $\Sigma$\\ \hline
    String over $\Sigma$  &	$u$, $v$, $w$ & A finite list of symbols from $\Sigma$\\ \hline
    The set of all strings over $\Sigma$ & $\Sigma^*$ & The collection of all possible strings formed from symbols from $\Sigma$ \\ \hline
    (Some) language over $\Sigma$& $L$ & (Some) set of strings over $\Sigma$ \\ \hline
    Empty string &$\varepsilon$ & The string of length $0$\\ \hline
    Empty set &$\emptyset$ & The empty language\\ \hline
    Natural numbers &$\mathcal{N}$ & The set of positive integers \\ \hline
    Finite set & & The empty set or a set whose distinct elements can be counted by a natural number\\ \hline
    Infinite set & & A set that is not finite.\\ 
    \hline\hline
    {\it Pages 3, 4, 13, 14 }& & \\
    \hline
    \end{tabular}
\end{center}

\newpage

\begin{center}
    \begin{tabular}{|p{2.7in}cp{3.8in}|}
    \hline
    Term & Notation & Meaning \\
    \hline \hline
    Reverse of a string $w$ & $w^\mathcal{R}$  & write $w$  in  the opposite order, if $w = w_1 \cdots  w_n$ then $w^\mathcal{R} = w_n \cdots  w_1$. Note: $\varepsilon^\mathcal{R} = \varepsilon$\\ \hline
    Concatenating strings $x$ and $y$ & $xy$ &  take $x = x_1 \cdots x_m$, $y=y_1 \cdots y_n$ and form $xy = x_1 \cdots x_m y_1 \cdots y_n$\\ \hline
    String $z$ is a substring of string $w$ & & there are strings $u,v$ such that $w = uzv$\\ \hline
    String $x$ is a prefix of string $y$ & & there is a string $z$ such that $y = xz$ \\ \hline
    String $x$ is a proper prefix of string $y$ & & $x$ is a prefix of $y$ and $x \neq y$\\ \hline
    Shortlex order, also known as string order over alphabet $\Sigma$ & & Order strings over  $\Sigma$ first by length and then according to the dictionary order, assuming symbols in $\Sigma$  have an ordering.\\ \hline
    \hline \hline
    {\it Pages 13, 14} & & \\
    \hline
    \end{tabular}
\end{center}

    

{\it Circle the correct choice}:

A {\bf string} over an alphabet $\Sigma$ is \underline{~~an element of $\Sigma^*$ ~~ OR ~~ a subset of $\Sigma^*$}.
    
A {\bf language} over an alphabet $\Sigma$ is \underline{~~an element of $\Sigma^*$ ~~ OR ~~ a subset of $\Sigma^*$}.


{\it Extra examples for practice:}

With $\Sigma_1 = \{0,1\}$ and $\Sigma_2 = \{a,b,c,d,e,f,g,h,i,j,k,l,m,n,o,p,q,r,s,t,u,v,w,x,y,z\}$  and $\Gamma = \{0,1,x,y,z\}$

An example of a string of length 3 over $\Sigma_1$ is \underline{\phantom{ $000$} \hspace{0.2in}}

An example of  a string of length 1 over $\Sigma_2$ is  \underline{\phantom{ $k$} \hspace{0.2in}}

The number of distinct strings of length 2 over $\Gamma$ is  \underline{\phantom{ $25$} \hspace{0.2in}}

An example of a language over $\Sigma_1$ of size $1$ is  \underline{\phantom{ $ \{ \varepsilon \} $} \hspace{0.2in}}

An example of an infinite language over $\Sigma_1$ is  \underline{\phantom{ $\Sigma^*$} \hspace{0.2in}}
    
An example of  a finite language over $\Gamma$ is  \underline{\phantom{ $\{ 0, x \}$} \hspace{0.2in}}
    
{\bf True} or {\bf False}: $\varepsilon \in \Sigma_1$

{\bf True} or {\bf False}: $\varepsilon$ is  a string over $\Sigma_1$

{\bf True} or {\bf False}: $\varepsilon$ is a language over $\Sigma_1$

{\bf True} or {\bf False}: $\varepsilon$ is a prefix of some string over  $\Sigma_1$

{\bf True} or {\bf False}: There is a string over $\Sigma_1$ that is a proper prefix of $\varepsilon$
    

The first five strings over $\Sigma_1$ in string order, using the ordering $0 <  1$: \vfill
    
The first five strings over $\Sigma_2$ in string order, using the usual alphabetical ordering for single letters: \vfill
