%! app: Decidable Languages, Undecidable Languages
%! outcome: Formal definition of automata, Informal definition of automata, Classify language, Find example languages



A {\bf Turing-recognizable} language is a set of strings that 
is the language recognized by some Turing machine. We also 
say that such languages are recognizable.

A {\bf Turing-decidable} language is a set of strings that 
is the language recognized by some decider. We also 
say that such languages are decidable.


An {\bf unrecognizable} language is a language that is not Turing-recognizable.

An {\bf undecidable} language is a language that is not Turing-decidable.

\vfill

{\bf  True} or {\bf False}: Any  decidable language  is  also  recognizable.

\vfill

{\bf  True} or {\bf False}: Any  recognizable language  is  also  decidable.

\vfill

{\bf  True} or {\bf False}: Any  undecidable language  is  also  unrecognizable.

\vfill

{\bf  True} or {\bf False}: Any  unrecognizable language  is  also  undecidable.

\vfill

\newpage


{\bf True} or {\bf False}: The class of Turing-decidable languages is closed under complementation.

\vfill

Using formal definition:
\vfill

Using high-level description:
\vfill


{\bf  Church-Turing Thesis} (Sipser p. 183): The informal notion of algorithm is formalized completely  and correctly by the 
formal definition of a  Turing machine. In other words: all reasonably expressive models of 
computation are equally expressive with the standard Turing machine.

\newpage
Definition: A language $L$ over an  alphabet $\Sigma$ is called {\bf co-recognizable} if its complement,  defined
as $\Sigma^* \setminus L  = \{ x  \in  \Sigma^* \mid x \notin  L \}$, is Turing-recognizable.


\vfill 
{\bf  Theorem} (Sipser Theorem 4.22): A  language is Turing-decidable if and only if both  it and its complement
are Turing-recognizable.

{\bf Proof, first direction:}  Suppose  language  $L$ is  Turing-decidable.   WTS  that both it and its complement 
are Turing-recognizable.

\vfill

{\bf Proof, second direction:}  Suppose  language  $L$ is  Turing-recognizable, and  so is  its complement.   WTS  that $L$
is Turing-decidable.
\vfill


Notation: The complement  of a set $X$ is denoted with  a superscript $c$, $X^c$, or an overline,  $\overline{X}$.

\newpage

{\bf Claim}: If two languages  (over a fixed alphabet  $\Sigma$) are Turing-decidable, then  their union  is  as well.

{\bf Proof}:


\vfill
\newpage

{\bf Claim}: If two languages  (over a fixed alphabet  $\Sigma$) are Turing-recognizable, then  their union  is  as well.

{\bf Proof}:

