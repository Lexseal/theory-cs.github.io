%! app: Decidable Languages, Undecidable Languages
%! outcome: Classify language, Classify decision problem, Reduction
    
{\bf Mapping reduction}

Motivation: Proving that $A_{TM}$ is undecidable was hard. How can we leverage that work? 
Can we relate the decidability / undecidability of one problem to another?

\begin{quote}
If problem $X$ is {\bf no harder than} problem $Y$

\ldots and if $Y$ is easy,

\ldots then $X$ must be easy too.
\end{quote}


\begin{quote}
    If problem $X$ is {\bf no harder than} problem $Y$
    
    \ldots and if $X$ is hard,
    
    \ldots then $Y$ must be hard too.
\end{quote}

``Problem $X$ is no harder than problem $Y$'' means 
``Can answer questions about membership in $X$ by converting them to questions about membership in $Y$''.



Definition:  $A$ is  {\bf  mapping  reducible to} $B$  means there is a computable function 
$f : \Sigma^* \to \Sigma^*$ such that {\it for all} strings  $x$ in $\Sigma^*$, 
\[
x  \in  A \qquad \qquad \text{if and  only  if} \qquad \qquad f(x) \in B.
\]
Notation:  when $A$  is mapping reducible to $B$, we write $A  \leq_m B$.

{\it Intuition:} $A \leq_m B$ means $A$ is no harder than $B$, i.e. that the level 
of difficulty of $A$ is less than or equal the level of difficulty of $B$.

\vfill

\newpage
{\bf Computable functions}

Definition: A function $f: \Sigma^* \to \Sigma^*$ is a {\bf computable function} means there is some Turing machine such that, 
for each $x$, on input $x$ the Turing machine halts with exactly $f(x)$ followed by all blanks on the tape

{\it Examples of computable functions}:

The function that maps a string to a string which is one character longer and 
whose value, when interpreted as a fixed-width binary representation of a
nonnegative integer is twice the value of the input string (when interpreted as 
a fixed-width binary representation of a non-negative integer)
\[
f_1: \Sigma^* \to \Sigma^* \qquad  f_1(x)  = x0
\]

To prove $f_1$ is computable function, we define a Turing machine computing it.

{\it High-level description}
\begin{quote}
    ``On input $w$
    
    1. Append $0$ to $w$.
    
    2. Halt.''
\end{quote}

{\it Implementation-level description}
\begin{quote}
    ``On input $w$
    
    1. Sweep read-write head to the right until find first blank cell.
    
    2. Write 0.
    
    3. Halt.''
\end{quote}

{\it Formal definition} $(\{q0, qacc, qrej\}, \{0,1\}, \{0,1,\textvisiblespace\},\delta, q0, qacc, qrej)$
where $\delta$ is specified by the state diagram: 


\newpage


The function that maps a string to the result of repeating the string twice.
\[
f_2: \Sigma^* \to \Sigma^* \qquad f_2( x )  =  xx
\]

\vfill

The function that maps strings that are not the codes of Turing machines to the empty 
string and that maps strings that code Turing machines to the code of the 
related Turing machine that acts like the Turing machine coded by the input, except
that if this Turing machine coded by the input tries to reject, the 
new machine will go into a loop.
\[
f_3: \Sigma^* \to \Sigma^*  \qquad f_3( x )  =   \begin{cases}  \varepsilon \qquad&\text{if $x$ is not the code of  a TM} \\
\langle (Q \cup \{q_{trap} \}, \Sigma, \Gamma, \delta', q_0, q_{acc}, q_{rej} ) \rangle \qquad&\text{if $x = \langle (Q, \Sigma, \Gamma, \delta, q_0, q_{acc}, q_{rej} )\rangle$}\end{cases}
\]
where $q_{trap} \notin Q$ and 
\[\delta'( (q,x) ) = \begin{cases}
(r,y,d) &\text{if $q \in Q$, $x \in \Gamma$, $\delta ((q,x)) = (r,y,d)$, and  $r \neq  q_{rej}$} \\
(q_{trap}, \textvisiblespace, R) & \text{otherwise}
\end{cases}
\]
\vfill

\newpage


The function that maps strings that are not the codes of CFGs to the empty 
string and that maps strings that code CFGs to the code of a PDA that recognizes
the language generated by the CFG.


\vfill

{\it Other examples?}
