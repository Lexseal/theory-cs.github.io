%! app: Decidable Languages, Undecidable Languages
%! outcome: Classify language, Classify decision problem, Reduction, Nondeterminism
 

In practice, computers (and Turing machines) don't have infinite tape, 
and we can't afford to wait unboundedly long for an answer.
``Decidable" isn't good enough - we want ``Efficiently decidable".

For a given algorithm working on a given input, how long do we need to wait for an answer? 
How does the running time depend on the input in the worst-case? average-case? 
We expect to have to spend more time on computations with larger inputs.


A language is {\bf recognizable} if \underline{\phantom{\hspace{4.5in}}}

A language is {\bf decidable} if \underline{\phantom{\hspace{4.7in}}}

A language is {\bf efficiently  decidable} if \underline{\phantom{\hspace{4in}}}

A function is {\bf computable} if \underline{\phantom{\hspace{4.7in}}}

A function is {\bf efficiently computable} if \underline{\phantom{\hspace{4in}}}\\

\vfill

Definition  (Sipser 7.1): For  $M$ a deterministic decider, its {\bf running time} is the function  $f: \mathbb{N} \to \mathbb{N}$
given  by
\[
f(n) =  \text{max number of  steps $M$ takes before halting, over all inputs  of length $n$}
\]

Definition (Sipser 7.7): For each function $t(n)$, the {\bf time complexity class}  $TIME(t(n))$, is defined  by
\[
TIME( t(n)) = \{ L \mid \text{$L$ is decidable by  a Turing machine with running time in  $O(t(n))$} \}
\]

An example of an element of  $TIME(  1  )$ is 

An example of an element of  $TIME(  n  )$ is 


Note: $TIME( 1) \subseteq TIME (n)  \subseteq TIME(n^2)$

\vfill

Definition (Sipser 7.12) : $P$ is the class of languages that  are decidable in polynomial time on 
a deterministic 1-tape  Turing  machine
\[
P  =  \bigcup_k TIME(n^k)
\]

{\it Compare to exponential time: brute-force search.}


Theorem (Sipser 7.8): Let $t(n)$ be a  function with  $t(n)  \geq n$.  Then every $t(n)$ time deterministic 
multitape Turing machine has an equivalent $O(t^2(n))$ time deterministic 1-tape Turing machine.



\newpage


Definition (Sipser  7.9): For $N$ a nodeterministic decider.  
The {\bf running time} of $N$ is the function $f: \mathbb{N} \to \mathbb{N}$ given  by
\[
f(n) =  \text{max number of  steps $N$ takes on  any branch before halting, over all inputs  of length $n$}
\]

Definition (Sipser 7.21): For each function $t(n)$, the {\bf nondeterministic time complexity class}  
$NTIME(t(n))$, is defined  by
\[
NTIME( t(n)) = \{ L \mid \text{$L$ is decidable by a nondeterministic Turing machine with running time in $O(t(n))$} \}
\]
\[
NP = \bigcup_k NTIME(n^k)
\]


{\bf True} or {\bf False}: $TIME(n^2) \subseteq NTIME(n^2)$

\vfill

{\bf True} or {\bf False}: $NTIME(n^2) \subseteq DTIME(n^2)$

\vfill

{\bf Examples in $P$ }

{\it Can't use nondeterminism; Can use multiple tapes; Often need to be “more clever” than naïve / brute force approach}
\[
    PATH = \{\langle G,s,t\rangle \mid \textrm{$G$ is digraph with $n$ nodes there is path from s to t}\}
\]
Use breadth first search to show in $P$
\[
    RELPRIME = \{ \langle x,y\rangle \mid \textrm{$x$ and $y$ are relatively prime integers}\}
\]
Use Euclidean Algorithm to show in $P$
\[
    L(G) = \{w \mid \textrm{$w$ is generated by $G$}\} 
\]
(where $G$ is a context-free grammar). Use dynamic programming to show in $P$.

\vfill
{\bf Examples in $NP$}

{\it ``Verifiable" i.e. NP, Can be decided by a nondeterministic TM in polynomial time,
best known deterministic solution may be brute-force, 
solution can be verified by a deterministic TM in polynomial time.}

\[
    HAMPATH = \{\langle G,s,t \rangle \mid \textrm{$G$ is digraph with $n$ nodes, there is path
from $s$ to $t$ that goes through every node exactly once}\}
\]
\[
    VERTEX-COVER = \{ \langle G,k\rangle \mid \textrm{$G$ is an undirected graph with $n$
nodes that has a $k$-node vertex cover}\}
\]
\[
    CLIQUE = \{ \langle G,k\rangle \mid \textrm{$G$ is an undirected graph with $n$ nodes that has a $k$-clique}\}
\]
\[
    SAT =\{ \langle X \rangle \mid \textrm{$X$ is a satisfiable Boolean formula with $n$ variables}\}
\]

\newpage